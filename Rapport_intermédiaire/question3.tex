%%Énoncé
Proposez un algorithme pour réaliser la méthode \textit{firstEntry()} (voir DSAJ-5
page 403 et DSAJ-6 page 396) dans un arbre binaire de recherche. Proposez un
algorithme pour réaliser la méthode \textit{higherEntry(k)} (voir DSAJ-5 page 403
et DSAJ-6 page 396) où \textit{k} est une clé présente dans un arbre binaire de recherche.

Quelles sont les complexités temporelles de vos algorithmes en fonction de \textit{h} (la
hauteur de l’arbre) et en fonction de \textit{n} (le nombre de noeuds de l’arbre) ?
(Tanguy)\footnote{Sources : \url{http://docs.oracle.com/javase/7/docs/api/java/util/TreeMap.html} et "TreeMap.java" issu du SDK 7}
\\
\\
%%Réponse
Algorithme pour \textit{firstEntry()}
\lstinputlisting[language=java, inputencoding=utf8]{firstEntryMethod.java}
Algorithme pour \textit{higherEntry(k)}
\lstinputlisting[language=java, inputencoding=utf8]{higherEntryMethod.java}

Pour les deux méthodes, la complexité temporelle est en $O (h)$ par rapport à la hauteur car il faut descendre tout en bas de l'arbre et en $O (\log (n))$ en fonction du nombre de noeuds car on ne tient compte que du sous-arbre de gauche à chaque itération.