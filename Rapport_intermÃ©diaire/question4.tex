Partant d’un arbre binaire de recherche initialement vide, comment se présente
l’arbre après y avoir inséré les clés 12, 5, 10, 3, 13, 14, 15, 17, 18, 15 ? Pour
les mêmes données comment se présenterait l’arbre finalement obtenu s’il s’agissait
d’un arbre (2,4) ? Pour les mêmes données comment se présenterait l’arbre
finalement obtenu s’il s’agissait d’un B-arbre d’ordre 4 ou d’un Splay Tree ? Cet
exemple illustre-t-il les avantages ou inconvénients de ces différentes structures
de données ? Pourquoi ? (Alexandre)

%%Réponse
\subsection*{Arbre binaire de recherche}
\begin{tikzpicture}
\node[circle,draw](z){$30$}
	child{
		node[circle,draw]{5} child{node[circle,draw] {3}} child{node[circle,draw] {10}}}
	child{
		node[circle,draw]{13} child[missing] 
			child{
				node[circle,draw]{14}  child[missing] child {
					node[circle,draw]{15}  child[missing] child{
						node[circle,draw]{17}   child{node[circle,draw] {15}}  child{node[circle,draw] {18}}
					}
				}
			}};
\end{tikzpicture}

\subsection*{Arbre (2,4)}
\begin{tikzpicture}
\tikzstyle{bplus}=[rectangle split, rectangle split horizontal,rectangle split ignore empty parts,draw]
\tikzstyle{every node}=[bplus]
\tikzstyle{level 1}=[sibling distance=20mm]
\node {10\nodepart{two} 13 \nodepart{three} 15}
  child {node {3 \nodepart{two} 5}}
  child {node {12}}
  child {node {14}} 
  child {node {16 \nodepart{two} 17 \nodepart{three} 18}};
  \end{tikzpicture}
  
  \subsection*{B-arbre d'ordre 4}
  \begin{tikzpicture}
\tikzstyle{bplus}=[rectangle split, rectangle split horizontal,rectangle split ignore empty parts,draw]
\tikzstyle{every node}=[bplus]
\tikzstyle{level 1}=[sibling distance=60mm]
\tikzstyle{level 2}=[sibling distance=15mm]
\node {12}
  child {node {5}
    child {node {3}}
    child {node {10}}
  } 
  child {node {14 \nodepart{two} 15}
    child {node {13}}
    child {node {15}}
    child {node {17 \nodepart{two}18}}    
  }
;\end{tikzpicture}
  \subsection*{Splay tree}
  \begin{tikzpicture}
  \tikzstyle{bplus}=[circle,draw]
  \tikzstyle{every node}=[bplus]
\node{$15$}
	child{node{15} child{node{14} child{node{13} child{node{5} child{node{3}} child{node{12} child{node{10}} child[missing]}} child[missing]} child[missing]} child[missing]}
	child{node{18} child{node{17}} child[missing]}
;\end{tikzpicture}

\subsection*{Conclusion}

Ces différents schémas permettent de mettre en évidence les différences qu'ils existent entre eux comme la profondeur de l'arbre, le nombre maximum d'élément fils d'un nœud...
  