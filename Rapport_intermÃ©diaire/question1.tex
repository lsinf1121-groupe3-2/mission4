%%Énoncé
\textbf{Les clés mémorisées dans les nœuds d'un arbre binaire de recherche peuvent-elles être autre chose que des nombres ? Si oui, donnez un exemple, sinon justifiez pourquoi. \\
Comment faire pour énumérer en ordre croissant toutes les clés mémorisées dans un arbre binaire de recherche ? Quelle est la complexité temporelle de cette opération ? \\
Si une même clé est mémorisée deux fois dans un arbre binaire de recherche, les nœuds correspondants sont-ils en relation père-fils ? Parfois, toujours, jamais ? Justifiez votre réponse.} (Boris)

%%Réponse
Les clés d'un arbre binaire de recherche pourraient être de n'importe quel type d'objet, tant que celui ci peut être comparé à d'autres objets du même type, c'est à dire que pour deux objets différents on puisse définir lequel des deux est supérieur à l'autre. En d'autres mots tant qu'une classe implémente l'interface Comparable, les objets de celle ci peuvent être utilisés comme clés dans un arbre binaire de recherche. Par exemple, on pourrait utiliser des Strings, et classer les éléments de l'arbre alphabétiquement.\\
On peut définir une fonction récursive qui liste toutes les clés mémorisées dans un arbre binaire de recherche en ordre croissant. Cette fonction discerne deux cas: si elle est appelée sur une feuille, elle ne fait rien, sinon, elle appelle la même fonction sur le sous arbre de gauche, puis elle imprime la clé de l'arbre courant, puis elle s'appelle sur le sous arbre de droite. La complexité temporelle de cette opération sera $O(n)$ n étant le nombre de d'éléments dans l'arbre, car elle effectuera n fois un nombre constant d'opérations.\\
Une clé identique ne peut jamais être mémorisée deux fois dans un même arbre de recherche binaire, car lorsqu'on veut ajouter un nœud dont la clé existe déjà dans l'arbre, la valeur de ce nœud sera modifiée sans ajouter de nœud. Si c'était quand même possible, ces deux nœuds devraient alors être en relation père fils, car sinon, il y aurait un nœud dont la clé se situe entre les deux nœuds de même  clé, ce qui est impossible car ils sont identiques.